\section{实验过程}

\subsection{Hamming 编解码}

编码器与解码器均采用组合逻辑编写,其中 
$$\mbox{生成矩阵}~\bm{G} = \begin{bmatrix}1 & 0 & 0 & 0 & 1 & 1 & 1\\ 0 & 1 & 0 & 0 & 1 & 0 & 1\\ 0 & 0 & 1 & 0 &1 & 1 & 0\\ 0 & 0 & 0 & 1 & 0 & 1 & 1\end{bmatrix},~ \mbox{校验矩阵}~\bm{H}^T=\begin{bmatrix}1 & 1 & 1\\ 1 & 0 & 1\\ 1 & 1 & 0\\ 0 & 1 & 1\\ 1 & 0 & 0\\ 0 & 1 & 0\\ 0 & 0 & 1\end{bmatrix}$$

于是编码器的Verilog关键代码为:

\begin{codeblock}{verilog}
assign data_o[0] = data_i[0];
assign data_o[1] = data_i[1];
assign data_o[2] = data_i[2];
assign data_o[3] = data_i[3];
assign data_o[4] = data_i[0] ^ data_i[1] ^ data_i[2];
assign data_o[5] = data_i[0] ^ data_i[2] ^ data_i[3];
assign data_o[6] = data_i[0] ^ data_i[1] ^ data_i[3];
\end{codeblock}

解码器的Verilog关键代码为:
\begin{codeblock}{verilog}
// 先计算校正子
wire [2:0] parity;
assign parity[0] = data_i[4] ^ data_i[0] ^ data_i[1] ^ data_i[2];
assign parity[1] = data_i[5] ^ data_i[0] ^ data_i[2] ^ data_i[3];
assign parity[2] = data_i[6] ^ data_i[0] ^ data_i[1] ^ data_i[3];
// 比较校正子和校验矩阵
assign data_o = (parity==3'b111) ? data_i[3:0]^4'b0001 : 
                (parity==3'b101) ? data_i[3:0]^4'b0010 :
                (parity==3'b011) ? data_i[3:0]^4'b0100 :
                (parity==3'b110) ? data_i[3:0]^4'b1000 :
                data_i[3:0];
\end{codeblock}


\subsection{交织与解交织}

交织器:当检测到 Hamming 码编码器成功完成 $m=4$ 次编码时,将 $mn=28$ 位编码器输出直接映射为 $28$ 位交织器输出。

Verilog 关键代码如下,通过采用直接映射的方式,有效提高了系统的效率。

\begin{codeblock}{verilog}
always @(posedge clk or negedge rst) begin
    if (~rst) begin
        data_o <= 0;
        eno <= 0;
    end
    else if (en) begin
        data_o[0] <= data_i[0];
        data_o[1] <= data_i[7];
        data_o[2] <= data_i[14];
        data_o[3] <= data_i[21];
        data_o[4] <= data_i[1];
        data_o[5] <= data_i[8];
        data_o[6] <= data_i[15];
        data_o[7] <= data_i[22];
        data_o[8] <= data_i[2];
        data_o[9] <= data_i[9];
        data_o[10] <= data_i[16];
        data_o[11] <= data_i[23];
        data_o[12] <= data_i[3];
        data_o[13] <= data_i[10];
        data_o[14] <= data_i[17];
        data_o[15] <= data_i[24];
        data_o[16] <= data_i[4];
        data_o[17] <= data_i[11];
        data_o[18] <= data_i[18];
        data_o[19] <= data_i[25];
        data_o[20] <= data_i[5];
        data_o[21] <= data_i[12];
        data_o[22] <= data_i[19];
        data_o[23] <= data_i[26];
        data_o[24] <= data_i[6];
        data_o[25] <= data_i[13];
        data_o[26] <= data_i[20];
        data_o[27] <= data_i[27];
        eno <= 1;
    end
end
\end{codeblock}

解交织器:当检测到 QPSK 解调器成功完成 $14$ 次解调时,将 $28$ 位解调器输出直接映射为 $28$ 位解交织器输出。

Verilog 关键代码如下,与交织过程类似。

\begin{codeblock}{verilog}
always @(posedge clk or negedge rst) begin
    if (~rst) begin
        r_data_o <= 0;
        r_eno <= 0;
    end
    else if (r_en) begin
        r_data_o[0] <= r_data_i[0];
        r_data_o[1] <= r_data_i[4];
        r_data_o[2] <= r_data_i[8];
        r_data_o[3] <= r_data_i[12];
        r_data_o[4] <= r_data_i[16];
        r_data_o[5] <= r_data_i[20];
        r_data_o[6] <= r_data_i[24];
        r_data_o[7] <= r_data_i[1];
        r_data_o[8] <= r_data_i[5];
        r_data_o[9] <= r_data_i[9];
        r_data_o[10] <= r_data_i[13];
        r_data_o[11] <= r_data_i[17];
        r_data_o[12] <= r_data_i[21];
        r_data_o[13] <= r_data_i[25];
        r_data_o[14] <= r_data_i[2];
        r_data_o[15] <= r_data_i[6];
        r_data_o[16] <= r_data_i[10];
        r_data_o[17] <= r_data_i[14];
        r_data_o[18] <= r_data_i[18];
        r_data_o[19] <= r_data_i[22];
        r_data_o[20] <= r_data_i[26];
        r_data_o[21] <= r_data_i[3];
        r_data_o[22] <= r_data_i[7];
        r_data_o[23] <= r_data_i[11];
        r_data_o[24] <= r_data_i[15];
        r_data_o[25] <= r_data_i[19];
        r_data_o[26] <= r_data_i[23];
        r_data_o[27] <= r_data_i[27];
        r_eno <= 1;
    end
end
\end{codeblock}

\subsection{QPSK 调制解调}

QPSK 调制器:

\begin{itemize} 
    \item QPSK 调制器根据输入的 2 比特数据,将其分别映射为 I 路和 Q 路的符号。
    \item 由于 QPSK 调制是基于正交载波的,调制器的输出信号对应于一个复数的实部和虚部,分别通过 I 路和 Q 路输出。
    \item 具体实现中,我们通过将每个输入符号映射到 $\pm 1/\sqrt{2}$ 乘以常数值来实现 QPSK 调制。常数值为 $1/\sqrt{2}$,其作用是对输出信号进行归一化处理,避免过强的信号造成干扰。
\end{itemize}

Verilog 关键代码如下:

\begin{codeblock}{verilog}
localparam signed [15:0] CONST_VAL = 16'b0101_1011;  // 1/sqrt(2)
always @(posedge clk or negedge rst) begin
    if (~rst) begin
        data_o_i <= 16'b0;  data_o_q <= 16'b0;
    end else begin
        case (data_i)
            2'b00: begin
                data_o_i <= CONST_VAL;  data_o_q <= CONST_VAL;
            end
            2'b01: begin
                data_o_i <= -CONST_VAL; data_o_q <= CONST_VAL;
            end
            2'b10: begin
                data_o_i <= CONST_VAL;  data_o_q <= -CONST_VAL;
            end
            2'b11: begin
                data_o_i <= -CONST_VAL; data_o_q <= -CONST_VAL;
            end
            default: begin
                data_o_i <= 16'b0;      data_o_q <= 16'b0;
            end
        endcase
    end
end
\end{codeblock}

QPSK 解调器:

\begin{itemize} 
    \item 解调器的作用是根据接收到的 I 路和 Q 路信号,判断信号属于哪一个象限,进而恢复原始数据。
    \item 解调器通过比较 I 路和 Q 路信号与阈值的大小关系,来判断比特的值。阈值设置为零,即信号大于零则为 1,小于零则为 0。
\end{itemize}

Verilog 关键代码如下:

\begin{codeblock}{verilog}
localparam signed [15:0] THRESHOLD = 16'b0;
always @(posedge clk or negedge rst) begin
    if (~rst) begin
        data_o <= 2'b00;
    end else begin
        if (data_i_i > THRESHOLD) begin
            if (data_i_q > THRESHOLD)   data_o <= 2'b00;
            else                        data_o <= 2'b10;
        end else begin
            if (data_i_q > THRESHOLD)   data_o <= 2'b01;
            else                        data_o <= 2'b11;
        end
    end
end
\end{codeblock}





\subsection{AWGN 信道模型}

\subsubsection{LFSR 生成均匀分布随机数}

在本实验中,为保证生成的两组随机数序列的独立性,LFSR 的伪随机数序列通过不同的特征多项式、不同的种子生成。具体地,选用不同的 SEED 值(即种子),并根据其最低有效位(LSB)选择特定的反馈多项式:

\begin{itemize} 
    \item SEED[0] = 1 时,特征多项式为 $X^{16} + X^{14} + X^{13} + X^{11} + 1$ 
    \item SEED[0] = 0 时,特征多项式为 $X^{16} + X^{15} + X^{13} + X^{4} + 1$ 
\end{itemize}

Verilog 代码实现如下:

\begin{codeblock}{verilog}
reg [15:0] lfsr_state;

always @(posedge clk or posedge rst) begin
    if (rst) begin
        lfsr_state <= SEED;
    end else begin
        if (SEED[0] == 1) begin
            lfsr_state <= {lfsr_state[14:0], lfsr_state[15] ^ lfsr_state[13] ^ lfsr_state[12] ^ lfsr_state[10]};
        end else begin
            lfsr_state <= {lfsr_state[14:0], lfsr_state[15] ^ lfsr_state[14] ^ lfsr_state[12] ^ lfsr_state[3]};
        end
    end
end
\end{codeblock}

在此过程中,生成的伪随机数序列被用于模拟 AWGN 信道中的噪声影响。为了确保生成的随机数适应特定范围的需求,采用如下代码进行范围调整:

\begin{codeblock}{verilog}
assign uniform_o = lfsr_state % (O_MAX - O_MIN + 1'b1) + O_MIN;
\end{codeblock}

该过程通过调整输出值的范围,确保生成的随机数序列符合信道噪声的要求。


\subsubsection{Box-Muller 生成高斯分布随机数}

为了提高系统的效率,我们采用查表法来计算 Box-Muller 变换中的 $\ln U_1$ 以及 $\cos(2 \pi U_2)$,以减少计算复杂度。具体来说,我们通过查表将 $U_1$ 和 $U_2$ 映射到预先计算好的值,从而实现快速变换。

特别地,对于 $\cos(2 \pi U_2)$,我们利用对称性和奇偶性,仅对 $U_1 \in [0, 0.25]$ 的部分进行查表映射,从而降低了存储需求并提高了查表效率:

\begin{codeblock}{verilog}
    wire cos_sign;
    assign cos_sign = (uniform_o_0 <= 256) ? 0 :
                      (uniform_o_0 <= 768) ? 1 :
                      0;
    wire [15:0] cos_addr;
    assign cos_addr = (uniform_o_0 <= 256) ? (uniform_o_0 - 1) :
                      (uniform_o_0 <= 512) ? (512 - uniform_o_0) :
                      (uniform_o_0 <= 768) ? (uniform_o_0 - 513) :
                      (1024 - uniform_o_0);
    wire signed [7:0] cos_lut_o;
    cos_lut cos_lut_0 (
        .addr(cos_addr),
        .cos_out(cos_lut_o)
    );

    wire signed [7:0] cos_o;
    assign cos_o = cos_sign ? -cos_lut_o : cos_lut_o;
\end{codeblock}

为了有效处理计算和存储,Box-Muller 变换的实现采用了定点数表示,通过算术左移和右移操作,在不溢出的情况下平衡了计算精度和速度。